%%%%%%%%%%%%%%%%%%%%%%%%%%%%%%%%%%%%%%%%%
% Rapport sur package R6
% Version 1 (14/11/2024)
%
% Auteurs :
% Souraya Ahmed Abderemane
% Lucile Perbet
% Béranger Thomas
%
%%%%%%%%%%%%%%%%%%%%%%%%%%%%%%%%%%%%%%%%%

%----------------------------------------------------------------------------------------
%	Packages
%----------------------------------------------------------------------------------------

% Classe du document
\documentclass[10pt,french]{report}

% Packages langue française
\usepackage[frenchb]{babel}
% Pour les guillemets en style français
\usepackage[french=guillemets]{csquotes}

% Saisir des caractères spéciaux et accentués
\usepackage[utf8]{inputenc}
% Imprimer des caractères spéciaux et accentués
\usepackage[T1]{fontenc}

% Améliorer les espacements
\usepackage{microtype}

\usepackage[a4paper, portrait]{geometry}

% Pour faire des tableaux plus efficacement
\usepackage{tabularray}

% Pour les symboles des équations mathématiques
\usepackage{mathtools}
\usepackage{amssymb}

% Pour charger et afficher des images
\usepackage{graphicx}
\graphicspath{{images/}}


% Pour représenter l'arborescence du programme
\usepackage{dirtree}


% Gérer les espaces entre paragraphes plus facilement
\usepackage{parskip}
\setlength\parindent{0pt}


% Polices de caractères
\usepackage{lmodern}
%\usepackage{stix2} % font stix 2
\usepackage{palatino} % font Palatino


% Pour créer des conditions dans les commandes customisées
\usepackage{ifthen}

\usepackage{amsmath}
\usepackage{longtable}


% Commande customisée pour dessiner un cube noir, pour faire séparateur
\newcommand{\cube}{\raisebox{0.13ex}{\scalebox{0.75}{ $\blacksquare$ }}}


% Commande customisée pour les entrées du lexique
\newcommand{\entreelex}[3][]{%
	{\large \textbf{\textsc{#2}}} % Entrée (obligatoire)
	\if\relax\detokenize{#1}\relax % Si #1 est vide
	\else % Si #1 n'est pas vide
	\raisebox{0.15ex}{\scalebox{0.7}{$\Diamond$}} % Diamant
	[#1] % Acronyme (facultatif)
	\fi
	\raisebox{0.13ex}{\scalebox{0.75}{$\blacksquare$}} #3 % Définition (obligatoire)
}


% Bibliographie : sous TexStudio, faire F5, puis F8, puis F5 à nouveau (c'est Latex, c'est tout simple)
\usepackage[style=numeric]{biblatex}
\addbibresource{Rapport.bib}


% Pour faire des hyperliens
\usepackage[colorlinks=true, linkcolor=NavyBlue]{hyperref}
% et leur donner une couleur plus sympa que le bleu habituel
\usepackage[dvipsnames]{xcolor}
\hypersetup{colorlinks=true, citecolor=NavyBlue}

%----------------------------------------------------------------------------------------

\begin{document}
	
	\begin{titlepage}
		\centering
		\includegraphics[width=0.6\textwidth]{icom.png}\par\vspace{0.5cm}
		{\Large\bfseries Université Lumière Lyon 2}\par
		\par\vspace{3.5cm}
		
		
		
		{\Huge\bfseries Application Friands}\par\vspace{0.75cm}
		{\huge\bfseries Finding Restaurants, Insights And Notably Delectable Spots}\par\vspace{0.75cm}
		
		{\Large\itshape }\par\vspace{3.5cm}
		
		{\Large\itshape Par}\par
		{\Large Souraya Ahmed Abderemane}\par
		{\Large Lucile Perbet}\par
		{\Large Béranger Thomas}\par\vspace{1cm}
		
		{\large\itshape Rapport présenté dans le cadre du}\par
		{\large Master 2 SISE - Promotion 2024/2025}\par\vspace{1cm}
		
		\vfill
		
	\end{titlepage}
	
	\tableofcontents
	
	\setlength{\parskip}{12pt}
	
	\begin{abstract}
		Ce rapport détaille l'implémentation d'une application Streamlit destinés aux lyonnais gourmets.
	\end{abstract}
	
	\chapter{Introduction}
	
	\chapter{Scraping}
	
	\section{Choix des outils}

	\subsection{API}
	
	\subsection{Beautifulsoup}
	
	\chapter{Base de données}
	
	\subsection{SQLite}
	
	Local
	Suffisant pour ce type d'appli
	Pas de soucis de rapidité
	Pas d'accès concurrents
	
	\subsection{Classe d'interfaçage}
	
	\subsection{Schéma}
	
	\chapter{Interface}

	\subsection{Streamlit}
	
	\subsection{Cartographie}
	
	\subsection{Ajout des restaurants}
	
	\subsection{Graphiques avec plotly}
	
	Interactifs, téléchargeables, rapides, etc
	
	\chapter{NLP - LLM}
	
	\subsection{API Mistral}
	
	Pour résumé
	
	\subsection{Spacy - NLTK}
	
	Etude inter-restaurants
	
	\subsection{Sentiment analysis}
	
	Modèles hugging faces - 5 étoiles pour comparer aux notes des gourmets.
	
	\chapter{Docker}
	
	\chapter{Augmentation de données}
	
	\subsection{Openstreetmap}
	
	\begin{itemize}
		\item restos alentour
		\item transports alentour
		\item coord gps
	\end{itemize}
	
	\chapter{Conclusion}
	
	La force de l'estomac solide soit avec vous.
	
	\end{document}
